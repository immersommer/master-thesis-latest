\chapter{EINLEITUNG}
\label{sec:intro}

Dieser Beleg beschreibt den Entwurf von anwendungsspezifischen integrierten Schaltungen (ASIC) im Modul "Schaltungs- und Systementwurf" an der Technischen Universität Dresden\cite*{Vorlesung}.

\vspace{\baselineskip}

\noindent Die in diesem Beleg implementierte Schaltung berechnet den größten gemeinsamen Teiler zweier Zahlen. Der größte gemeinsame Teiler ist ein mathematischer Begriff für die größte positive ganze Zahl, die mehr als eine ganze Zahl teilen kann. Und mehrere ganze Zahlen können nicht alle Null sein. Für den Entwurf wird der euklidische Algorithmus verwendet. Dieser Algorithmus ist der älteste der aktuellen Algorithmen. Dieser Algorithmus wurde ursprünglich nur für natürliche Zahlen und geometrische Längen verwendet. Später wurde der Algorithmus auf andere Bereiche der Mathematik wie die Knotentheorie und multivariate Polynome ausgedehnt. Im hier vorliegendem Beleg wird der Algorithmus zur Berechnung des größten gemeinsamen Teilers verwendet.

\vspace{\baselineskip}

\noindent Zur Entwicklung der Schaltung wurden LogicFriday zur Erstellung der Logikgleichungen und des
Schaltungsentwurfs für den Zustandsautomaten sowie die Entwicklungsumgebung Cadence
Design Framework2 zur Simulation und Implementierung der Schaltung eingesetzt.

\vspace{\baselineskip}

\noindent In Kapitel 2 dieses Belegs wird der Prozess des Algorithmus beschrieben. Kapitel 3 beschreibt den gesamten Entwurfsprozess und die Ideen bei der Entwicklung der Schaltung. In Kapitel 4 werden Simulationen durchgeführt, um die Korrektheit der Funktionalität des Algorithmus zu überprüfen.

\cleardoublepage

