\chapter{ZUSAMMENFASSUNG UND WERTUNG}
\label{sec:evaluation}

Die Simulationsergebnisse können beweisen, dass die Entwurfsziele erreicht wurden. Die gewählte Architektur ermöglicht es, den Algorithmus schnell und effizient zu implementieren. Darüber hinaus wird der Hardwareaufwand minimiert und effektiv genutzt.

\vspace{\baselineskip}

\noindent Es ist wichtig zu beachten, dass die Eingaben in der Simulation nicht dezimal, sondern hexadezimal sind. Die Addition und der Speicher sollten entsprechend den Anforderungen ausgewählt werden, und es sollte ein geeignetes Modell gewählt werden. Wir verwenden hier einen 64-Bit-Addition.

\vspace{\baselineskip}

\noindent Diese Entwurfsarbeit hat mir eine allgemeine Übersicht darüber verschafft, wozu ich als Entwickler in der Lage sein sollte. Ein Entwickler muss die Hardware-Implementierung schon zu Beginn des Entwurfs berücksichtigen. Bei meinem ersten Design traten Probleme bei der Simulation auf, und ich habe viele Veränderungen vorgenommen, bevor ich es erfolgreich umsetzen konnte.

\vspace{\baselineskip}

\noindent Ich möchte mich auch bei meinem Praktikumsbetreuer bedanken. Er stand mir immer zur Verfügung, um meine Fragen zu beantworten und mir Anregungen zu geben, die mir halfen, meinen Entwurf schneller fertigzustellen.
\cleardoublepage

