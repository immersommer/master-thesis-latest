\chapter{Evaluation}
\label{sec:evaluation}

% Zu jeder Arbeit in unserem Bereich gehört eine Leistungsbewertung. Aus
% diesem Kapitel sollte hervorgehen, welche Methoden angewandt worden,
% die Leistungsfähigkeit zu bewerten und welche Ergabnisse dabei erzielt
% wurden. Wichtig ist es, dem Leser nicht nur ein paar Zahlen
% hinzustellen, sondern auch eine Diskussion der Ergebnisse
% vorzunehmen. Es wird empfohlen zunächst die eigenen Erwartungen
% bezüglich der Ergebnisse zu erläutern und anschließend eventuell
% festgestellte Abweichungen zu erklären.
In previous chapters, we analyzed the security issues of the vanilla quark and proposed the design and implementation of the confidential Quark as a countermeasure. In this chapter, we evaluate the security and performance of Confidential Quark from both 
qualitative and quantitative perspectives.

\section{Qualitative Analysis}
In this section, we evaluate the proposed mitigations for the vulnerabilities found in Section III.

\subsection{Common Setup}


\begin{figure}[!htb] 
    \begin{subfigure}[b]{0.45\linewidth}
      \centering
      \includegraphics[width=0.9\linewidth]{images/generic_policy.PNG} 
      \caption{Enclave policy} 
      \label{fig:generic_policy} 
      \vspace{4ex}
    \end{subfigure}%% 
    \begin{subfigure}[b]{0.45\linewidth}
      \centering
      \includegraphics[width=0.9\linewidth]{images/analysis_workload.png} 
      \caption{Program used for the qualitative analysis} 
      \label{fig:analysis_workload} 
      \vspace{4ex}
    \end{subfigure} 
    \caption{Artifacts for demo in the qualitative analysis}
    \label{fig4} 
\end{figure}




This section outlines the configuration for the analysis. First, we set up the Confidential Quark and Vanilla Quark environments following the guidelines provided in the Quark demo repository~\cite*{Qaurk_Demo_for_qualitativ}. The policy used by Confidential Quark is shown in Figure~\ref{fig:generic_policy}. The policy specifies that the Enclave is in production mode. 
Privileged users can allocate terminals and issue cat and ls commands, while non-privileged level users can only issue ls commands. Privileged-level commands can be executed in the directory /var and its subdirectories, while non-privileged-level commands can only access the directory /var/log and its subdirectories. In addition, the Qlog manager is turned on. 
Since allowed\_max\_log\_level is set to OFF, Qkernel does not print any logs. Finally, the guest system call interceptor is activated and runs in ContextBase mode. The system call allowlist contains all system calls (0-451), some of which are not included in the figure due to space limitations.


Figure~\ref{fig:generic_policy} shows the example program used for the quantitative analysis. The program first print the environment variable and application parameter type secrets. It then accesses the secret "root-ca.pem" under directory /secret. The application-related secret is shown in Figure~\ref{fig5}. We use script~\cite*{secret_uploading_script} to deploy these secrets and the enclave 
policy to the secret manager. Note that the application is already containerized. The artifacts for the demo in the qualitative analysis and the YAML files for Deploying the demo can be found in commit~\cite*{artifacts_quarlitative}.

\begin{figure}[!htb] 
    \begin{subfigure}[b]{0.45\linewidth}
      \centering
      \includegraphics[width=0.9\linewidth]{images/file_secrets.PNG} 
      \caption{File type secrets managed by Secret Manager} 
      \label{fig:file_secrets} 
      \vspace{4ex}
    \end{subfigure}%% 
    \begin{subfigure}[b]{0.45\linewidth}
      \centering
      \includegraphics[width=0.9\linewidth]{images/cmd_env_secrets.png} 
      \caption{Args and Envv based secrets managed by Secret Manager} 
      \label{fig:cmd_env_secrets} 
      \vspace{4ex}
    \end{subfigure} 
    \caption{Secrets for demo in the quantitative analysis}
    \label{fig5} 
\end{figure}


\subsection{Application secure Deployment}
\begin{figure}[!htb]
    \centering
    \includegraphics[height=0.3\textheight]{images/cquark_deploy_yaml.jpg}
    \caption[YAML for deploying an application in confidential Quark]{YAML for deploying an application in confidential Quark}
    \label{fig:cquark_deploy_yaml}
\end{figure}

Confidential Quark deploys the application using YAML, as shown in Figure~\ref{fig:cquark_deploy_yaml}. The new application deployment scheme and YAML are explained in Section~\ref{sec:secure_application_deployment}. Figure~\ref{fig:cquark_deployment} outlines the steps and outcomes of the application deployment process in the Confidential Quark environment.

Figure~\ref{fig:cquark_kbs_start} shows that the enclave successfully attests its identity to the secret manager during remote attestation and provisioning. Currently, the secret manager operates in native mode, verifying the consistency between the enclave start hash and the reference hash specified in the 
enclave owner's policy. The enclave's startup hash includes the measurement for application binaries, Qkernel configuration files, etc. (refer to Section III). As depicted in Figure~\ref{fig:cquark_deployment_result}, the enclave successfully retrieves the secret from the secret manager and prints them. It's worth 
noting that the directory named "secret" remains invisible from the host (as seen in Figure~\ref{fig:cquark_deployment_result_file_secret_mount_location}), as file type secrets and the filesystem mounted under the /secret directory are within the enclave。 Therefore, security issues~\ref{vulnerabilities:1},~\ref{vulnerabilities:7}, and~\ref{vulnerabilities:11} 
have been effectively resolved.


\begin{figure}[!htb]
    \subfloat[Secret manager succesfully autheticate enclave's state during remote attestation\label{fig:cquark_kbs_start}]{%
    \includegraphics[clip,width=\columnwidth]{images/cquark_kbs_start.PNG}%
  }

    \subfloat[Application deployment process in confidential quark\label{fig:cquark_deployment_result}]{%
      \includegraphics[clip,width=\columnwidth]{images/cquark_deployment_result.PNG}%
    }
    
    \subfloat[The way enclave manages the file type secrets\label{fig:cquark_deployment_result_file_secret_mount_location}]{%
      \includegraphics[clip,width=\columnwidth]{images/cquark_deployment_result_file_secret_mount_location.png}%
    }
    
    \caption[The result for application secure deployment in confidential Quark]{The result for application secure deployment in confidential Quark.\label{fig:cquark_deployment}}
\end{figure}





\subsection{Authentication and Access Control for EXEC Requests}


% \begin{figure}[H]

%     \subfloat[Accessing application log using securectl in confidential quark\label{fig:cquark_log_from_secure_ctl}]{%
%       \includegraphics[clip,width=\columnwidth]{images/cquark_log_from_secure_ctl.png}%
%     }
    
%     \subfloat[Accessing application log using kubectl in confidential quark\label{fig:cquark_log_from_kubectl}]{%
%       \includegraphics[clip,width=\columnwidth]{images/cquark_log_from_kubectl.png}%
%     }
    
%     \caption[Accessing application log through kubectl and securectl in confidential quark]{Accessing application log through kubectl and securectl in confidential quark. Figures 1 and 2 show that privileged users in cquark, i.e., 
%     application owners, have normal access to the application logs, while non\-privileged users using kubectl logs only obtain encrypted garbled code.}
% \end{figure}
\begin{figure}[!htb]

    \subfloat[The enclave rejects the privileged and non-privileged command ”ls /“"\label{fig:ls_exec}]{%
    \includegraphics[clip,width=\columnwidth]{images/ls_exec.png}%
    }

    \subfloat[Issuing unprivileged command "ls /var/log"\label{fig:ls_allow_unprivileged}]{%
    \includegraphics[clip,width=\columnwidth]{images/ls_allow_unprivileged.png}%
    }

    \subfloat[Issuing privileged command "ls /var/"\label{fig:ls_allow_previleged}]{%
      \includegraphics[clip,width=\columnwidth]{images/ls_allow_previleged.png}%
    }
    
    \subfloat[Issuing privileged command "cat" \label{fig:cat_pri}]{%
      \includegraphics[clip,width=\columnwidth]{images/cat_pri.png}%
    }

    \subfloat[Issuing privileged command "cat" \label{fig:cat_unpri}]{%
    \includegraphics[clip,width=\columnwidth]{images/cat_unpri.png}%
  }

  \subfloat[Attempting to allocate a terminal through both securectl and kubectl\label{fig:cquark_terminal}]{%
  \includegraphics[clip,width=\columnwidth]{images/cquark_terminal.png}%
}
    
    \caption[The result of issuing  commands to enclave]{The result of issuing  commands to enclave\label{fig:exec}}
\end{figure}


In Section~\ref*{sec:design_EXEC_Requests}, we propose a novel schema for handling EXEC requests. As depicted in Figure~\ref{fig:ls_exec}, both privileged and unprivileged users are prohibited from executing the "ls /" command within the enclave due to the enclave policy's restriction on running commands in the root directory. 
Figure~\ref{fig:ls_allow_unprivileged} and~\ref{fig:ls_allow_previleged} illustrate that privileged and unprivileged users can issue the "ls /var" and "ls /var/log" commands to the enclave, respectively. This allowance stems from the enclave policy's command and directory allowlist, which includes these specific commands and their corresponding arguments. 
Conversely, since the command whitelist for non-privileged users excludes the "cat" command, unprivileged users are incapable of issuing the "cat" command to access the log file "/var/log/fontconfig.log."(figure~\ref{fig:cat_unpri}) 
Furthermore, as observed in Figure~\ref{fig:cquark_terminal},  privileged users can spawn a terminal, while unprivileged users cannot. 
This is as expected since the unprivileged user's command allowlist does not include any terminal allocation instructions. By implementing this innovative mechanism, we effectively safeguard the enclave against attacks exploiting the EXEC endpoint, resolving weakness three as identified in our security analysis.


\begin{figure}[!htb]
    \centering
    \includegraphics[width=1\textwidth]{images/cquark_priviled_cmd_result_protection.PNG}
    \caption[Process for securectl to handle privileged level commands]{Process for securectl to handle privileged level commands.}
    \label{fig:cquark_priviled_cmd_result_protection}
\end{figure}



In addition, we modified securectl to print the command it issues, and the execution result returned from the enclave. As can be seen in Figure~\ref{fig:cquark_priviled_cmd_result_protection}, both the EXEC command and its execution result are cryptographically protected. The protection mechanisms for the command and the execution result are 
discussed in Sections~\ref{sec:design_prptect_privileged_request} and~\ref{sec:design_STDIO_PROTECTION}, respectively.


This section shows that we can effectively address vulnerability~\ref{vulnerabilities:4} and ~\ref{vulnerabilities:6} identified in the security analysis.



\subsection{Guest User Space Process's STDIO Protection}
The application shown in Figure~\ref{fig:analysis_workload} is non-interactive. Since the no\_interactive\_process\_stdout\_err\_encryption option in the policy is true, the enclave will encrypt STDOUT and STDERR for privilege-level processes. Thus, only privileged users, namely the application owners, can view the logs using securectl (as shown in Figure~\ref{fig:cquark_log_from_secure_ctl}). On the other hand, non-privileged users who use the command 
"kubectl log" can only access the logs in ciphertext (as depicted in Figure~\ref{fig:cquark_log_from_kubectl}). Furthermore, Figure 4 showcases that the execution results of privilege-level commands are also encrypted.

\begin{figure}[!htb]

    \subfloat[Accessing application log using securectl\label{fig:cquark_log_from_secure_ctl}]{%
      \includegraphics[clip,width=\columnwidth]{images/cquark_log_from_secure_ctl.png}%
    }
    
    \subfloat[Accessing application log using kubectl\label{fig:cquark_log_from_kubectl}]{%
      \includegraphics[clip,width=\columnwidth]{images/cquark_log_from_kubectl.png}%
    }
    
    \caption[Accessing non-interactive application log through kubectl and securectl]{Accessing non-interactive application log through kubectl and securectl}
\end{figure}

In the case of interactive applications, we use the busybox as an example. The corresponding YAML file for deployment can be found in commit~\cite*{artifacts_busybox}. Currently, the enclave lacks support for a terminal shielding layer. Nevertheless, we set the interactive\_porcess\_stdout\_err\_encryption option in the policy to true. 
This way, the enclave encrypts the STDOUT and STDERR outputs for the interactive application process. Figure~\ref{fig:attach_failed} illustrates the scenario where a non-privileged user can connect a terminal to the application. However, the terminal functionality is limited as the enclave only returns ciphertexts. 
Furthermore, the logs generated by the interactive application are also encrypted (Figure~\ref{fig:interactive_log}). These examples effectively showcase how our 
the proposed solution successfully mitigates vulnerabilities~\ref{vulnerabilities:5} and~\ref{vulnerabilities:3}.


\begin{figure}[!htb]

    \subfloat[Attaching to application using kubectl -it attach\label{fig:attach_failed}]{%
    \includegraphics[clip,width=\columnwidth]{images/attach_failed.png}%
  }

  \subfloat[Accessing application log using kubectl\label{fig:interactive_log}]{%
  \includegraphics[clip,width=\columnwidth]{images/interactive_log.png}%
}
  
    \caption[Accessing logs and attaching to an interactive application ]{Accessing logs and attaching to an interactive application }
\end{figure}



\subsection{Qkernel Log Management}

In Vanilla Quark, the logs generated by Qvsivor and Qkernel are kept in the same file, /var/log/quark/quark.log. In order to differentiate the logs, we prefix the logs generated by Qkernel with the keyword "Qkernel" (refer to Figure~\ref{fig:vanilla_qkernel_Log}).
\begin{figure}[!htb]

    \subfloat[Vanilla Quark\label{fig:vanilla_qkernel_Log}]{%
    \includegraphics[clip,width=\columnwidth]{images/vanilla_qkernel_Log.png}%
  }

  \subfloat[Confidential Quark\label{fig:cquark_qkernel_log}]{%
  \includegraphics[clip,width=\columnwidth]{images/cquark_qkernel_log.png}%
}
  
    \caption[Accessing guest kernel log in confidential and vanilla Quark]{Accessing guest kernel log in confidential and vanilla Quark}
\end{figure}


In Confidential Quark, we set the Qlog manager's max log level to OFF. Therefore, the guest kernel logging system does not print logs to the host (Figure~\ref{fig:cquark_qkernel_log}).

\subsection{Qkernel Guest System Call Interception}

\begin{figure}[!htb]
    \centering
    \includegraphics[width=1\textwidth]{images/application_failed_to_start_due_to_syscall_interceptor.png}
    \caption[Failed to run an application when the read system call is excluded from the system call interceptor's allowed list]{Failed to run an application when the read system call is excluded from the system call interceptor's allowed list}
    \label{fig:application_failed_to_start_due_to_syscall_interceptor}
\end{figure}


Figure ~\ref{fig:application_failed_to_start_due_to_syscall_interceptor} presents a demo of the guest interceptor. By revising the setting of the system interceptor in the enclave policy of Figure ~\ref{fig:generic_policy} to remove the read system call from the system call allowlist, 
the application process failed to start. The failure occurred because it lacked permission to execute the read system call for loading dynamic libraries.

\subsection{Application Runtime Measurements}




According to the discussion in Section~\ref{sec:Enclave_Runtime_Measurement}, the enclave measures the binaries loaded from the host at application runtime and compares the resulting hashes with the reference hashes in the enclave policy. Binary refers to shared libraries and executables. 

Figure~\ref{fig:cquark_runtime_runtime_lib_measurement_demo} shows a measurement demo of the runtime library. Regarding the setup, we intentionally reset the reference hash of the library "/lib/x86\_64-linux-gnu/libgcc\_s.so.1" in the enclave policy to empty and configure the maximum logging level of the Qlog manager to Info. Next, 
we deploy the policy to the secret manager and use the demo application in Figure~\ref{fig:analysis_workload}. The Qkernel logs in Figure~\ref{fig:cquark_runtime_runtime_lib_measurement_demo} show that the enclave encountered a crash due to a discrepancy between the measurements of the loaded library and the corresponding reference hash stored in the policy.


\begin{figure}[!htb]
    \centering
    \includegraphics[width=1\textwidth]{images/cquark_runtime_runtime_lib_measurement_demo.png}
    \caption[Shared library measurement demo]{Shared library measurement demo}
    \label{fig:cquark_runtime_runtime_lib_measurement_demo}
\end{figure}




To demonstrate measuring the executable loaded during application runtime, we intentionally configured the enclave policy to set the reference hash for the "ls" binary to empty and the Qlog manager's max log level to Info. Then, we submit the new policy to the secret manager,  
redeploy the application, and issue the command "ls /var/log" to the enclave. Figure~\ref{fig:cquark_runtime_runtime_binary_measurement_demo}  shows that the command failed to get a response. By analyzing the Qkernel logs, we confirmed that the reason for the Qkernel crash was a mismatch between the reference hash of the ls binary and its actual hash.


\begin{figure}[!htb]
    \centering
    \includegraphics[width=1\textwidth]{images/cquark_runtime_runtime_binary_measurement_demo.png}
    \caption[Runtime binary measurement demo]{Runtime binary measurement demo}
    \label{fig:cquark_runtime_runtime_binary_measurement_demo}
\end{figure}



The above two examples highlight the ability of an enclave to refuse execution when an adversary tries to trick the enclave into loading harmful code during application runtime.

\subsection{Missing administration to application restartup}


As discussed in Section~\ref{sec:Enclave_Runtime_Measurement}, the enclave will measure the loaded binaries and shared libraries during the application rebuild process to extend the application restart hash. The enclave can ensure the rebuild process is correct by comparing the 
application startup reference hash to the restart hash.
\begin{figure}[!htb]

    \subfloat[Application restart workflow\label{fig:app_restart_eva}]{%
    \includegraphics[clip,width=\columnwidth]{images/app_restart_eva.png}%
  }

  \subfloat[Qkernel log\label{fig:app_restart_result}]{%
  \includegraphics[clip,width=\columnwidth]{images/app_restart_result.png}%
}
  
    \caption[Demo for  administration to application restartup]{Demo for  administration to application restartup}
\end{figure}



The application restarting process is exemplified in Figure~\ref{fig:app_restart_eva}. In this example, we configure the Qlog manager’s max log level to ’Info’ in the enclave policy and add the keywords ’stdin: true’ and ’tty: true’ to the container spec of the YAML file used to deploy the 
application (Figure~\ref{fig:cquark_deployment}). These keywords keep the application’s stdin open and notify Kubernetes that it should be regarded as a terminal. We first create the application in the demo using the modified YAML file. Once the application starts, we use ’kubectl attach -it’ to 
attach to the application process and send ’ctrl c’ to the enclave. The ctrl c will cause the terminal IO redirection thread sends SIGINT to terminate the application, as discussed in section~\ref{sec:security_analyse_STDIO}. The output of the second ’kubectl get’ command confirmed that the application had successfully restarted. 

After analyzing the Qkernel log shown in Figure~\ref{fig:app_restart_result}, we found that Qkernel measured binaries and shared libraries loaded from the host during the application restart. The enclave compared the resulting measurements with the application startup reference hash to determine that the binaries and shared libraries loaded during the two startups were 
consistent.

\subsection{Vulnerability Out of Scope}
\todo{move to discussion}
Besides the vulnerabilities already discussed, there are other security risks that affect Vanilla Quark but are beyond the scope of this thesis.



\textbf{Physical Access Attacks.} An untrustworthy entity, including cloud operators who have physical access to the machine where the application runs, can carry out physical access attacks on the system. One example of such attacks 
is Offline DRAM Analysis. A partial solution to alleviate the threat is to run guests on AMD SEV SNP or INTEL TDX.  Those interested in learning more about this topic should refer to \cite*{SEV_SNP_white_book}, \cite*{DBLP:journals/corr/abs-2303-15540}.


\textbf{Lack of protection to guest memory and register states.} Lack of protection to guest memory and register states.  Insufficient protection for guest memory and register states allows non-trusted entities, such as a hypervisor, to access guest secrets, leading to security breaches. 
Trusted execution hardware like AMD SEV\cite*{SEV_SNP_white_book} and Intel TDX\cite*{Intel_tdx_whitepaper} address this by encrypting guest memory and registers. However, CQuark does not yet support running on the Trusted execution hardware yet.


\textbf{Paravirtualized filesystem sharing mechanism.} Vanilla Quark uses this mechanism to share files and volumes stored on the host with applications on the guest. Any modifications made by an application on a file will be reflected on the host. This poses a security risk as an attacker could 
gain access to the application’s private data by accessing the application's rootfs on the host. A common approach is implementing a guest filesystem shielding layer to protect inbound and outbound guest data. Those interested in learning more about this topic should 
refer to \cite*{file_system_shield}.


\textbf{No security guarantee for communications over the Internet (unbounded network access to containers).} The qkernel in Vanilla Quark lacks authentication to network requests and cryptographic protection for transmitted data, resulting in potential unauthorized access to the application or 
man-in-the-middle attack\cite*{Man_in_the_middle_attack}. For instance, attackers can use Kubectl port forwarding to connect to an application listening on the target port and acquire the application’s confidential data or service. A possible solution to this is implementing a guest network shield that leverages TLS to authorize 
incoming network requests and provides cryptographic protection for incoming and outgoing data. However, this thesis does not cover this type of vulnerability, and readers are encouraged to refer to reference \cite*{network_shiled}.

\textbf{Side channel attack.} Side channels, such as TLB side channel\cite*{217454}, cache side channel\cite*{7163050}, ciphertext side channel\cite*{274707} , and page-fault side-channel\cite*{236278}, can jeopardize the confidentiality of programs.  This type of vulnerability is considered out of the scope.



% \begin{table}[H]
%     \tiny
%     \caption{Possible Vulnerability in upstream Quark in the context of Confidential Computing and suggested Solutions}
%     \label{crouch}
%     \begin{tabular}{  p{3.4cm}  p{3.4cm}  p{3.4cm} p{2cm} }
%         \toprule
% \textbf{Vulnerability}      
% & \textbf{Attack Example}   
% & \textbf{Possible Solution}
% & \textbf{Comment}  \\\midrule
% Physical Access Attacks
% & Offline DRAM Analysis          
% & Running the application in a secure virtual machine
% & Out of the scope   \\\hline
% Lack of protection to guest memory and register states          
% & Hypervisor reads private guest memory/cpu states                          
% & Running the application in a secure virtual machine
% & Out of the scope   \\\hline
% Paravirtualized filesystem sharing mechanism       
% & Hypervisor/Untrusted host process reads application’s credentials stored on host
% & Enable file system shielding layer to encrypt outbound and decrypt inbound data 
% & Out of the scope   \\\hline

% No security guarantee for communications over the Internet (unbounded network access to containers)     
% & Attacker access application’s credential by establishing a network connection to application using tools like, kubectl port-forward
% & Enable network shielding layer using TLS
% & Out of the scope   \\\hline

% Deploying Secrets via untrusted Entities (Kubelet, Containerd, Qvisor)    
% & File type secrets are mounted to container rootfs by qvisor, args and envv type secrets are passed to qkernel through qvisor
% & Offload the secrets deployment from qvisor by defining a new secure channel btw. relying party and guest kernel for secrets provisioning and storing secrets on guest memory
% & Problem solved  \\\hline

% Executing arbitrary command in container  
% & Attacker view application’s credentials stored on guest memory using kubectl exec cat command
% & Enable authentication and access control to kubectl exec in guest kernel
% & Problem solved  \\\hline


% Container log in plaintext managed by untrusted entity
% & Attacker reads application’s log message using kubectl logs container-name
% & Enable container STDOUT protection
% & Problem solved  \\\hline

% Storing Qkernel Log on host in plaintext
% & Cloud provider reads guest kernel’s log messages located in directory /var/log/quark
% & Enable qkernel log manager and set the log level to OFF
% & Problem solved  \\\hline

% Loading untrusted executable from host
% & Attackers may tamper with executables stored on the host and trick applications into executing compromised code to reveal secrets
% & Executable loaded into guest memory is measured and the results are send to relying party for executable integrity check
% & Problem solved  \\\hline

% Loading untrusted shared library from host
% & Attackers may tamper with shared libraries stored on the host and trick applications into executing compromised code to reveal secrets
% & Executable loaded into guest memory is measured and the results are send to relying party for executable integrity check
% & Problem solved  \\\hline

% Missing administration to application restart
% & Attacker may provide the guest with compromised executables, shared libraries, or wrong process spec when k8s requests the Qkernel to restart the crashed application
% & Compare the hash of the application rebuilding process with the hash of application's initial launching process stored on guest memory. If two hashes doesn’t match,  qkernel refuse the application restart request.
% & Problem solved  \\\hline


% No restriction to Container's syscalls (guest system calls)
% & Applications can be tricked into using vulnerable guest/host system calls, leading to disclosure of secrets
% & Using Guest system call interceptor to restrict the system calls application can use.
% & Problem solved  \\\hline


% Creating application process using untrusted process spec sent from host
% & Attacker may trick the qkernel into attaching a terminal to application process by modifying the "terminal" option in the process specification
% & Software measurement manager measure the loaded process specification and the results are send to relying party for integrity check
% & Problem solved  \\\hline


% Lack of runtime measurements
% & Secure VM like AMD SEV only calculate the hash of VM launching process, anything loaded to guest during runtime is not measured
% & Add software measurement manager to measure data loaded during runtime
% & Problem solved  \\\hline
% % objects and systems &
% % Underlying values       
% % & Plurality \\
%         \bottomrule
%     \end{tabular}
% \end{table}


\section{Quantitative Analysis}
In this section, we evaluate the performance of our system implementation in terms of both latency and throughput.

\subsection{Overview of Quantitative Analysis}
\todo{correct grammar error and revise}
Our benchmarks  consist of micro-benchmarks and macro-benchmarks. The microbenchmarks are performed to  thoroughly understand the latency overhead of each new building block in CQuark, while the macro benchmarks are conducted to compare the speed of 
real world applications run by confidential vs vanilla Quark. This give us a view of the overhead incurred by confidential Quark for protecting application’s confidentiality.


The benchmarking in this section is structured around the following subsections. First, we describe the hardware and software setup for benchmarking in Section \ref{Hardware_and_Software_Setup}. We then evaluate the speed of obtaining three different types of attestation reports (emulated hardware 
reports, software reports signed by the kbs key or by the application key) using a microbenchmark that executes a certain number of a system call in a loop and records the time required to execute the system call (Section \ref{Attestation_Report_Syscall}). We found that the speed of obtaining the 
emulated hardware report is 50\% slow than that of obtaining the other two reports due to VM exit. 

Subsequently, in Section \ref{bench_Interceptor}, we evaluate the overhead of the guest system call interceptor using a micro test that executes a certain number of a specific system call in a loop in vanilla and confidential Quark, respectively. The results show that the guest system interceptor imposes an overhead of 1.47x 
and 17x for system calls that are handled only by the guest and system calls that must be handled by the host, respectively

Unlike vanilla Quark, which stores file-type secrets on the host, cquark stores file-type secrets directly on guest memory. Therefore, in Section \ref{bench_reading_file_secret}, we evaluate the speeds of an application accessing file-type secrets in vanilla and confidential Quark using a similar 
approach as in Section \ref{Attestation_Report_Syscall}. The results reveal that in confidential Quark, the speed of accessing file-type secrets is improved by a factor of 1.17.

% the authentication and access control for the exec endpointin 
Later, in Section \ref{bench_issuing_Instructions}, we present latency  benchmark results for sending commands to containers in confidential and vanilla Quark. Compared to the speed of sending commands using kubectl exec\cite*{kubectl} in vanilla quark, confidential Quark's implementation 
introduces 1.15 times overhead. Furthermore, We compare the speed of sending the same command with kubectl and securectl in confidential Quark. The results show that despite the additional encryption and decryption required by the commands sent by securectl, the commands are 
executed 1.1 times faster than those sent by kubectl.

Then after conducting an evaluation of the performance of accessing binary-mapped memory regions in cquark, we observed that confidential quark outperforms vanilla quark in this regard (Section \ref{accesiing_binary_mapped_memory}). Sequential access in confidential quark is 1.86 times faster than in 
vanilla quark, while random reading and writing of binary-mapped memory regions exhibit approximately 1.08 times faster speed in confidential quark. These results can be attributed to confidential quark's measurement of binary files during program startup, which prompts the qkernel to load 
the files into memory, thus preventing  page fault handling during runtime.

In Section \ref{micro_app_start_up}, we conducted an evaluation of the latency overhead introduced by the our implementation during application deployment using a "hello world" program.  Specifically, the evaluation involved testing the latency overhead of the attestation and provisioning agent, 
secret injector, software measurement manager, and hardware evidence driver. Throughout the experiments, we adjusted a critical parameter to observe how each component affects the overall application startup time. Findings indicate that the attestation and provisioning agent constitutes 
the bulk of the overhead when the size of the measurement data is less than 100 MB, while the overhead introduced by the secret injector and hardware evidence driver is negligible in any case. 
% Further details, including the benchmark configuration and results can be found on page XX.

In Section \ref{macri_app_start_up}, we conclude the performance benchmarking exercise by conducting macro benchmarking using Nginx \cite*{nginx} and Redis \cite*{redis} as workloads. Our benchmarks focus on exploring the performance differences between applications running in confidential and vanilla Quark 
in terms of startup time, exit time, and the application throughput. Our findings reveal that in confidential Quark, Redis \cite*{redis} and Nginx \cite*{nginx} require 1.49X and 1.55X startup times than in vanilla quark, respectively. The difference of startup time between Nginx and Redis in 
confidential Quark primarily arises from the contrast in data size measured during application startup, i.e., 67.15 Mib in Nginx and 16.258 MiB in Redis. In terms of application exit time, Redis and Nginx exhibit nearly identical performance when running on confidential  and vanilla quark. 
Furthermore, we utilized Redis-benchmark \cite*{Redis_benchmark} and Apache HTTP server benchmarking tool \cite*{ab} as load generators to gauge the application's throughput. Our experiment results indicate that Nginx and Redis running in cquark encounter performance declines of approximately 
9\% and 20\%,respectively, due to guest system interceptor.

In Section \ref{tcb}, we evaluated the TCB overhead incurred by our implementation and compared the TCB size of the Confidential Container \cite*{confidential_kata} with that of the confidential Quark. Our findings reveal that our implementation increases the quark guest kernel binariy by approximately 1.47 times. Nevertheless, 
the TCB of confidential Quark is only roughly one-sixteenth of the size of the  Confidential Container \cite*{confidential_kata}.

\subsection{Hardware and Software Setup}\label{Hardware_and_Software_Setup}

This section summarizes the hardware and software settings utilized for the benchmarks. Regarding hardware, all measurements are performed on a server with AMD EPYC 7443P 24-Core CPU and 4 DDR4-3200 Mhz-16GB RAM. Regarding software, the host OS is Ubuntu 20.04.6 LTS and Linux 5.19.0 kernel. 
Furthermore, the CPU governor is set to performance mode, and Hyper-Threading and Turbo Boost were disabled to establish a low-noise environment for benchmarking with reproducible results. Besides, all test binaries, namely Qvisor, Qkernel, and the ”hello-world” program in List~\ref{code1}, are built in 
release mode. Unless otherwise mentioned, the vanilla Quark is in version v0.2.0, and confidential Quarkuses use this commit~\cite*{qualitativ_confidentail_quark}.


Regarding the tools for measuring latency, since all benchmark objects are located in the guest, we use the guest system call clock\_gettime(monotonic)~\cite*{clock_gettime} for time measurement and print the results to the host using the Qkernel logging system. The 
benchmarking framework~\cite*{benchamark_framework} can analyze the results by parsing the Quark log under the host directory /var/log/Quark. In addition, all benchmarks are performed multiple times. The following sections illustrate the measurement results in average time and variance format.



\subsection{Micro-benchmark – Attestation Report Syscall}\label{Attestation_Report_Syscall}

Confidential Quark enables applications to obtain three types of attestation reports via a system call. The type of the report includes simulated AMD SEV SNP report, software reports signed by KBS key, or software reports signed by the application. This section assesses the speed with which the 
application acquires various reports. For this purpose, we prepared a Rust microbenchmark\cite*{benchamark_Attestation_Report_Syscall}. In order to ensure stable results, the test program submitted 10,000 requests for each type of report to the Qkernel. The benchmark program is containerized, 
and the YAML file for the deployment can be found in the commit~\cite*{perf_attestation_report}.

\begin{figure}[!htb]
    \centering
    \includegraphics[width=0.5\textwidth]{images/perf_attestation_report_result.PNG}
    \caption[Benchmark result of Attestation Report Syscall]{Qkernel Attestation Report Syscall Benchmark Result}
    \label{fig:perf_attestation_report_result}
\end{figure}

Figure~\ref{fig:perf_attestation_report_result}  depicts the result. The data reveal that requesting a simulated hardware report takes approximately 838.141us, whereas acquiring the two software reports needs only about 286us. This difference stems from the fact that generating the hardware report involves expensive operations 
such as virtual machine exit, loading dummy SNP report from the disk, etc. Consequently, obtaining a simulated hardware report is significantly slower than a software report.


\subsection{Micro-benchmark – Qkernel Syscall Interceptor}\label{bench_Interceptor}

Interception guest system calls introduces latency overhead to applications running in the confidential Quark.  To investigate this overhead, we employ a micro benchmark program written in Rust\cite*{benchamark_systemcall_intercetion}.  The benchmark measures the average and variance execution 
time of chosen system calls. Furthermore, to obtain stable results., each system call is executed in a loop of 100,000 times. As the benchmark program is containerized, we can deploy the program to Cquark and upstream Quark using the YAML file in Figure \ref{fig:syscall_interceptor_yaml}.
\begin{figure}[H]
    \centering
    \includegraphics[width=0.8\textwidth]{images/perf_system_call_interceptor_yaml.PNG}
    \caption[Qkernel Syscall Interceptor Benchmark Deployment]{Qkernel Syscall Interceptor Benchmark Deployment}
    \label{fig:syscall_interceptor_yaml}
\end{figure}


For this benchmark, we select four system calls: getpid, getppid, read, and write. Since qkernel already implements process and thread objects, calling getpid or getppid using syscall instruction prompts qkernel to directly return the corresponding pid/ppid without interference from the host. 
This provides us with a clear understanding of the overhead introduced by the interceptor. Additionally, we measured the execution time of read and write system calls for confidential and vanilla Quark to evaluate the overhead that the interceptor brought to the host-handled system calls
\begin{figure}[H]
    \centering
    \includegraphics[width=0.8\textwidth]{images/ben_results_syscall_interceptor.PNG}
    \caption[Benchmark result of Syscall Interceptor]{Latency comparison of executing syctem calls in Cquark vs upstream Qaurk. The latency is thereby measured using the guest system call clock\_gettime(monotonic). 
        Each system call is executed 1,000,000 times in a tight loop. Read and write buffer size is 100 bytes}
    \label{fig:ben_results_syscall_interceptor}
\end{figure}


Figure ~\ref{fig:ben_results_syscall_interceptor} illustrates the mean and variance of execution times for the selected system calls in the both environments. We observed that vanilla Quark takes only around 270 ns to execute a single getpid and getppid call while confidential Quark
takes approximately 4716 ns. This implies that the guest system call interceptor causes a delay of about 4500 ns per system call. Moreover, a comparable latency overhead is introduced for read and write system calls, i.e.,4500 ns.  However, since read and write system calls are processed on 
the host side and take longer, their processing time in confidential Quark does not rise as significantly as that of getpid and getppid. To summarize, in confidential Quark, host handled system calls  (i.e., write and read) and guest handled-only (e.g., getpid and getppid) are 1.47x and 
17x slower, respectively, compared to vanilla Quark


The overhead incurred by the system call interceptor arises from two primary sources. Firstly, the whitelist of the guest system call interceptor is kept in a vector. Accessing the whitelist for system call authorization check is an operation with the complexity of O(N). 
Secondly, to avert any contentions resulting from run-time updates and reads to the system call interceptor's metadata, this metadata is safeguarded with a lock. Consequently, a lock/unlock operation is initiated for every system call permission check.



\subsection{Micro-benchmark – Speed of Reading file base secret}\label{bench_reading_file_secret}

Compared to vanilla Quark which stores file type secrets on the host, confidential Quark instead stores them on guest memory to prevent unauthorized access and avoid potential VM exits during read and write operations. Hence, the purpose of this benchmark is to evaluate the performance 
enhancement achieved from accessing file type secrets in confidential Quark. The benchmark\cite*{benchamark_filebase_secret} measures the total time taken to perform one open() and read() operation on a target secret type file in both environments. The file size is 1302 bytes. To ensure stable results, 10,000 open and 
read operations are performed. As the benchmark program is containerized, we deploy it to both confidential and vanilla Quark using the YAML file outlined in Figure ~\ref{fig:file_type_secret_access_test_deploy_yaml_baseline} and ~\ref{fig:file_type_secret_access_test_deploy_yaml_cquark}.


% \begin{figure}[H]
%     \centering
%     \subfloat[\centering Deployment File for upstream Quark]{{\includegraphics[width=5cm]{images/file_type_secret_access_test_deploy_yaml_baseline.PNG} }}
%     \qquad
%     \subfloat[\centering Deployment File for confidential Quark]{{\includegraphics[width=5cm]{images/file_type_secret_access_test_deploy_yaml_cquark.PNG} }}
%     \caption{Benchmark Deployment for testing Speed of Reading file type Secrets }
%     \label{fig:file_type_secret_access_test_deploy_yaml_baseline}
% \end{figure}
% \end{document}

\begin{figure}
    \centering
    \begin{minipage}{0.45\textwidth}
        \centering
        \includegraphics[width=0.9\textwidth]{images/file_type_secret_access_test_deploy_yaml_baseline.PNG} % first figure itself
        \caption{Benchmark Deployment for testing Speed of accessing file type Secrets in upstream Quark}
        \label{fig:file_type_secret_access_test_deploy_yaml_baseline}
    \end{minipage}\hfill
    \begin{minipage}{0.45\textwidth}
        \centering
        \includegraphics[width=0.9\textwidth]{images/file_type_secret_access_test_deploy_yaml_cquark.PNG} % second figure itself
        \caption{Benchmark Deployment for testing Speed of accessing file type Secrets in confidential Quark}
        \label{fig:file_type_secret_access_test_deploy_yaml_cquark}
    \end{minipage}
\end{figure}
\todo{update Benchmark Deployment for testing Speed of accessing file type Secrets in upstream Quark, change it to 100000}


\begin{figure}[H]
    \centering
    \includegraphics[width=0.8\textwidth]{images/reading_speed_of_file_type_secrets_in_Baseline_and_Cquark.PNG}
    \caption[Benchmark result of testing Speed of accessing file type Secrets in confidential Quark vs upstream quark]{The figure shows result of testing speed of accessing file type secrets in confidential vs vanilla Quark. 
    The evaluations is performed using the open() and read() system calls, and ten thousand (10,000) operations are conducted with a 50-byte read buffer size.
    }
    \label{fig:reading_speed_of_file_type_secrets_in_Baseline_and_Cquark}
\end{figure}


The obtained results are presented in Figure ~\ref{fig:reading_speed_of_file_type_secrets_in_Baseline_and_Cquark}, revealing that the performance  in confidential Quark is competitive with vanilla Quark for accessing file-type secrets. Surprisingly, there is no significant difference in speed
between the two, with confidential Quark providing only 1.17 times the performance improvement when accessing the target secret file.  The reason is that the larger performance loss resulting from system call execution due to the client system call interceptor in confidential Quark outweighs the overhead produced by VM exits in vanilla Quark.


\subsection{Micro-benchmark – Latency Test for issuing Instructions to Application}\label{bench_issuing_Instructions}

This section presents the results of tests conducted to assess the latency of issuing instructions to applications in confidential Quark. The tests involved comparing the latency of sending commands using kubectl between confidential and vanilla Quark, as well as the speed comparison test for issuing unprivileged 
and privileged commands in confidential Quark using kubectl and securectl respectively.


To facilitate the tests, we extended our test framework to enable it to issue specific instructions to the application using either kubectl or securectl, while recording command execution time\cite*{benchamark_perf_kubectl}. Similar to previous benchmarks, each instruction was iterated 1,000 times 
to ensure dependable results. Due to time constraints, we were unable to measure all available commands on the Linux system, so we chose four commonly used commands for the following testing: pwd, ls, cat, and cp.

% \begin{figure}
%     \centering
%     \begin{minipage}{0.45\textwidth}
%         \centering
%         \includegraphics[width=0.9\textwidth]{images/speed_of_issuing_cmd_in_cquark_upstream_quark.PNG} % first figure itself
%         \caption{Benchmark Deployment for testing Speed of accessing file type Secrets in upstream Quark}
%         \label{fig:speed_of_issuing_cmd_in_cquark_upstream_quark}
%     \end{minipage}\hfill
%     \begin{minipage}{0.45\textwidth}
%         \centering
%         \includegraphics[width=0.9\textwidth]{images/timeshare_issuing_cmd_in_cquark_upstream_quark_kubectl.PNG} % second figure itself
%         \caption{Benchmark Deployment for testing Speed of accessing file type Secrets in confidential Quark}
%         \label{fig:timeshare_issuing_cmd_in_cquark_upstream_quark_kubectl}
%     \end{minipage}
% \end{figure}

\begin{figure}[H]
    \centering
    \includegraphics[width=0.8\textwidth]{images/speed_of_issuing_cmd_in_cquark_upstream_quark.PNG}
    \caption[Benchmark result - Latency Comparison of sending Commands to Application in Cquark vs vanilla Quark]{A latency comparison test of sending commands to applications in cquark vs vanilla quark.  The test selects four 
    commonly used Linux commands as samples, each being executed one thousand times. The results indicate that utilizing Kubectl exec to issue instructions to applications in cquark takes over 9\% longer than in vanilla quark.
    }
    \label{fig:speed_of_issuing_cmd_in_cquark_upstream_quark}
\end{figure}




\begin{figure}[H]
    \centering
    \includegraphics[width=0.8\textwidth]{images/timeshare_issuing_cmd_in_cquark_upstream_quark_kubectl.PNG}
    \caption[Benchmark result - The processing time for each component during the instruction execution lifecycle in confidential vs in vanilla Quark]{Benchmark result of evaluating the percentage of time spent by each component in processing instructions during the instruction execution lifecycle 
    in confidential vs in vanilla Quark . Here the components refer to kubectl,  Kubernetes components for instructions transmission, like containerd, etc., and Qkernel for instruction execution. 
    Each bar in the figure represents the total execution time of an instruction whereby the color-filled section represents the time qkernel used for instruction execution, while the dashed-filled segment implies the time other components took for instruction issuing, transmission, and result handling.
    }
    \label{fig:timeshare_issuing_cmd_in_cquark_upstream_quark_kubectl}
\end{figure}


This section analyses the results illustrated in Figures ~\ref{fig:speed_of_issuing_cmd_in_cquark_upstream_quark} and ~\ref{fig:timeshare_issuing_cmd_in_cquark_upstream_quark_kubectl}. Our finding show that confidential Quark introduces approximately 9\% overhead to the instruction 
execution compared to vanilla quark. This overhead stems from several factors. Firstly, unlike upstream quarks, the Qkernel in Cquark performs authentication and access control for each instruction. Besides, the measurement of the instruction's corresponding binary and guest system call 
interception increases the instruction execution time.  This argument is confirmed in Figure 5.1b, where the average time for Qkernel to execute instructions in vanilla quark is between 1 and 2 ms, while in Cquark, instructions execution in Qkernel takes more than 10 ms. Thus, issuing 
instructions using Kubectl in confidential Quark proves to be more expensive than in vanilla quark.

\begin{figure}[H]
    \centering
    \includegraphics[width=0.8\textwidth]{images/speed_of_issuing_cmd_in_cquark_kubctl_securectl.PNG}
    \caption[Benchmark result - Latency Comparison of issuing privileged instruction vs unprivileged instruction in confidential Quark]{A latency comparison test of issuing privileged instruction vs unprivileged instruction in confidential Quark.  The test included four frequently used Linux 
    commands as samples, each being executed one thousand times.The results indicate that issuing privileged instructions is faster than issuing unprivileged instruction.
    }
    \label{fig:speed_of_issuing_cmd_in_cquark_kubctl_securectl}
\end{figure}


Out of curiosity, we compared the time required to issue privileged and unprivileged instructions in confidential Quark to the application using Securectl and Kubectl, respectively. The results in Figure ~\ref{fig:speed_of_issuing_cmd_in_cquark_kubctl_securectl} reveal that issuing privileged 
instructions is much faster than issuing unprivileged commands. This finding is counterintuitive since privileged instructions require additional cryptographic protection to ensure the confidentiality and integrity of their contents, which implies executing additional code.

\begin{figure}[H]
    \centering
    \includegraphics[width=0.8\textwidth]{images/timeshare_issuing_cmd_in_cquark_kubectl_securectl.PNG}
    \caption[Benchmark result - The processing time for components during the privileged vs unprivileged instruction execution lifecycle]{Benchmark result evaluating the percentage of time spent by each component in processing instructions during the privileged versus unprivileged instruction 
    execution lifecycle . Here the components refer to the tool the user employed to issue instructions, i.e., kubectl or securecli,  Kubernetes components for instructions transmission, like containerd, etc., and Qkernel for instruction execution.  Each bar in the 
    figure represents the total execution time of an instruction whereby the color-filled section represents the time qkernel used for instruction execution, while the dashed-filled segment implies the time other components took for instruction issuing, transmission, and result handling.
    }
    \label{fig:timeshare_issuing_cmd_in_cquark_kubectl_securectl}
\end{figure}

To investigate the issue further, we conducted  measurements of processing time for each component during the privileged vs unprivileged instruction execution lifecycle. Here the components refer to the tool the user employed 
to issue instructions, i.e., kubectl or securecli,  Kubernetes components for instructions transmission, like containerd, etc., and Qkernel for instruction execution.
Figure ~\ref{fig:timeshare_issuing_cmd_in_cquark_kubectl_securectl} presents the results obtained. The time incurred by qkernel for executing unprivileged and privileged instructions (color-filled boxes) distinctly shows that the latter takes approximately 0.5 milliseconds more than the former does. 
These latency overheads derive from the decryption and integrity checks that kernel performs for each privileged instruction, as well as the cryptographic protection for the instruction result. However, comparing the time taken by kubectl and securectl to process user requests 
(dashed filled boxes), securectl is significantly faster. Thus, this offsets the cryptographic protection overhead for privileged instructions.

\subsection{Micro-benchmark – Latency Test for accessing binary-mapped memory region}\label{accesiing_binary_mapped_memory}

This section presents the results of a microbenchmark that assesses the speed of accessing binary mapped memory regions during runtime under vanilla and confidential Quark.
% //%or \small or \footnotesize etc.
% \begin{lstlisting}[language=C,frame=single,caption=Program for testing speed of accessing binary-mapped memory region Variant,label=code3,basicstyle=\tiny]
%     #define ARRAY_LEN (unsigned long)(1024*1024*100)    
%     char array[ARRAY_LEN] = {[ 0 ... (1) ] = 'a'} ;

%     int main {
%         struct timespec start, end;
%         clockid_t clk_id = CLOCK_MONOTONIC;  // CLOCK_REALTIME CLOCK_BOOTTIME CLOCK_PROCESS_CPUTIME_ID
%         struct timespec nanos;
%         clock_gettime(CLOCK_MONOTONIC, &nanos);
%         srand(nanos.tv_nsec); 
    
%         for(int i = 0; i < ARRAY_LEN; i++) {
%            index[i] = get_random();
     
%         }
     
%         clock_gettime(clk_id, &start);
%         for(int i = 0; i < ARRAY_LEN; i++) {
%            int current_idx = index[i];
%            //sequential read
%            // char a = array[i];
%            //sequential write
%            // array[i] = 'b';
%            //random read
%            // char a = array[current_idx];
%            //random write
%            // array[current_idx] = 'b';
%         }
%         clock_gettime(clk_id, &end);
     
%         unsigned long long  start_ns = to_ns(start);
%         unsigned long long  end_ns = to_ns(end);
%         printf("start %ju, end  %jun", start_ns, end_ns);

%         return 0;

%     }

 
% \end{lstlisting}

The benchmark utilizes a program\cite*{benchamark_micro} containing a global initialized array located in the binary data segment and a for loop. Within the for loop, the program can access the array by means of four access mode: random read, random write, sequential read, and sequential write. 
We conduct the test in three binary data segment mapped memory sizes, i.e. 5MB, 10MB and 100MB. For each case, the test framework carries out 30 executions of the program in vanilla and confidential Quark in each access mode and calculates the average cost of accessing the array. Note that the size of the binary data 
segment mapped memory region is controlled by the size of the array.


\begin{figure}[ht] 
    \begin{subfigure}[b]{0.5\linewidth}
      \centering
      \includegraphics[width=0.9\linewidth]{images/Sequential_Read.PNG} 
      \caption{Sequential Read} 
      \label{fig7:a} 
      \vspace{4ex}
    \end{subfigure}%% 
    \begin{subfigure}[b]{0.5\linewidth}
      \centering
      \includegraphics[width=0.9\linewidth]{images/Sequential_Write.PNG} 
      \caption{Sequential Write} 
      \label{fig7:b} 
      \vspace{4ex}
    \end{subfigure} 
    \begin{subfigure}[b]{0.5\linewidth}
      \centering
      \includegraphics[width=0.9\linewidth]{images/Random_Read.PNG} 
      \caption{Random Read} 
      \label{fig7:c} 
    \end{subfigure}%%
    \begin{subfigure}[b]{0.5\linewidth}
      \centering
      \includegraphics[width=0.9\linewidth]{images/Random_Write.PNG} 
      \caption{Random Write} 
      \label{fig7:d} 
    \end{subfigure} 
    \caption{Benchmark Result - Latency Test for accessing binary-mapped memory region}
    \label{fig7} 
\end{figure}



These results shown in Figure ~\ref{fig7:a} and ~\ref{fig7:b} reveal that confidential Quark improves the performance of sequentially reading and writing binary-mapped memory regions. Specifically, Sequential accessing in confidential Quark is 1.86 times faster than in vanilla quark. This performance 
improvement is due to the fact that measuring binary loaded from host during the program startup prompted qkernel to load the program's binaries from disk into the guest memory. Consequently, the program can access these memory areas at runtime without necessitating page fault handling. 
For the same reason, random reading and writing of binary-mapped memory regions exhibit around 1.08 times faster speed in confidential Quark than in vanilla Quark (Figure ~\ref{fig7:c} and Figure ~\ref{fig7:d}).



\subsection{Micro-benchmark – Latency Test for Application Startup}\label{micro_app_start_up}
The introduction of remote attestation, secret provisioning, and data measurement loaded from the host in confidential Quark results in additional latency overhead for application startup. To examine this overhead, we extended the testing framework to iteratively initiate and terminate the 
application, measuring the start time, exit time, and latency overhead incurred by the remote attestation and provisioning agent, secret injector, hardware evidence driver, and software measurement manager\cite*{benchamark_framework}.

\begin{figure}[H]
    \centering
    \includegraphics[width=0.8\textwidth]{images/micro_benchmark_app_life_cycle_bencmark_pattern.png}
    \caption[Benchmark Pattern - Latency Test for Application Startup]{Benchmark Pattern - Latency Test for Application Startup}
    \label{fig:micro_benchmark_app_life_cycle_bencmark_pattern}
\end{figure}


Regarding the benchmark methodology. A statically linked hello-world program in Listing ~\ref{code1} is used. To reduce the interference of environmental noise,the test framework runs the application 100 times. In addition, since the remote attestation and provisioning agent, the secret injector, 
the hardware evidence driver, and the software measurement manager shown in Figure  ~\ref{fig:micro_benchmark_app_life_cycle_bencmark_pattern}  (colored boxes) are located in the guest kernel, the test framework cannot directly measure the time spent by these components. Therefore, we use guest clock\_gettime(monotonic)\cite*{clock_gettime} to 
record the time and print the latency overhead of each component to the host using the qkernel logging system. The same approach applies to record the size of data measured by the software measurement manager. 

\begin{lstlisting}[language=C,frame=single,caption=Hello World Program,label=code1]
int main() {
    printf("Hello, World!\n");
    return 0;
}
\end{lstlisting}


As expounded in Chapter 4, two factors may impact the application startup time in confidential Quark: the number of file type secrets and the size of the data measured by the software measurement manager. Hence, this section conducts following three tests. First, we establish a baseline by setting the 
number of file-type secrets to zero and running the hello world program. The test framework measures the program startup time, and records the latency overhead from the remote attestation and provisioning agent, the secret injector, the hardware evidence driver, and the software measurement manager。  
In Experiment 2, we increase the number of file type secrets in order to observe the change in program startup time, as well as the latency overhead from the remote attestation and provisioning agent and the secret injector. In Experiment 3, we examine the growth in the latency overhead of the software 
measurement manager during program startup as the amount of measured data increased. To conduct the test, we utilize a variant of the hello-world program as shown in Listing ~\ref{code2}, incorporating a global initialization array to regulate the data segment size in the binary and thus modify the size 
of software manager measured data during program startup.

\begin{lstlisting}[language=C,frame=single,caption=Hello World Program Variant,label=code2]
#define ARRAY_LEN (unsigned long)(1024*1024*100)    
char array[ARRAY_LEN] = {[ 0 ... (1) ] = 'a'} ;
int main() {     
    printf("Hello, World!\n");
    return 0;
}
\end{lstlisting}


\subsubsection{Creating a Baseline}

This benchmark measures the hello world program's startup time, and the latency overhead caused by remote attestation and provisioning agent, the secret injector, the hardware evidence driver, and the software measurement manager. Note that here the number of file type secrets is 0.

\begin{figure}[H]
    \centering
    \includegraphics[width=0.8\textwidth]{images/application_start_microtest_baseline_time_overhead_each_cmp.PNG}
    \caption[Benchmark results: Latency Overhead introduce by new Componnents in Cquark]{Benchmark results: Latency Overhead introduce by new Componnents in Cquark}
    \label{fig:application_start_microtest_baseline_time_overhead_each_cmp}
\end{figure}

\begin{figure}[H]
    \centering
    \includegraphics[width=0.8\textwidth]{images/application_start_microtest_baseline_time_distribution.PNG}
    \caption[Time  Distribution of the application startup ]{Distribution of the time consumed by different components in the application startup}
    \label{fig:application_start_microtest_baseline_time_distribution}
\end{figure}



The benchmark results in figure ~\ref{fig:application_start_microtest_baseline_time_overhead_each_cmp} and ~\ref{fig:application_start_microtest_baseline_time_distribution} shows that the remote attestation and provisioning agent, the secret injector, the hardware evidence driver, 
and the software measurement manager doubled the program startup time. Notably, the remote attestation and provisioning agent contributed the most to the latency overhead, which is expected, given its necessity to establish the TLS connection with the relying party, complete the remote 
attestation process, and fetch the enclave policy following the KBS attestation protocol. With regards to the secret injector, the hardware evidence object driver, which is primarily responsible for loading secrets into the application process and generating a simulated hardware report accordingly,  
exhibited a delay overhead of 0.90 milliseconds and 1.73 milliseconds, respectively. However, this delay is negligible compared to the overhead caused by the remote attestation and provisioning agent (~837 milliseconds).  Furthermore, based on the data presented in 
Table ~\ref{table: Measurement_For_Hello_world}, the software measurement manager is found to measure approximately 1.47 MiB of data during program startup, leading to an overhead of only about 10 milliseconds.


\begin{table}[htbp]
    \centering
    \footnotesize
    \caption{Software measurement manager measured data during application startup\strut}
    \begin{tabularx}{1\textwidth}{@{} l L L L L L L@{}}
    \toprule
        Matrics    & ELF &  Shared Library   & Process Spec    & Guest Kernel Args  & Total Measurement\\
    \midrule

        \vn{Measurement}     &  1.447 MiB    &   0 MiB  &   304 Byte      &25004 Byte  & 1.47`' MiB\\
    \bottomrule
    \end{tabularx}
    \label{table: Measurement_For_Hello_world}
\end{table}



\subsubsection{Determining Attestation \&\& Provisioning Agent and Secret Injector Overhead}
The baseline results confirm that the remote attestation and provisioning agent is a significant source of latency overhead. To this end, this section aims to examine how the latency overhead of the remote attestation and provisioning agent, along with the secret injector, changes as the number 
of file-type secrets increases.

\begin{figure}[H]
    \centering
    \includegraphics[width=0.8\textwidth]{images/overhead_attestation_agent_as_file_num_increasing.PNG}
    \caption[Benchmark result: Latency Overhead from Attestation \&\& Provisioning Agent as the number of file-based secrets increases]{Benchmark result: Latency Overhead from Attestation \&\& Provisioning Agent as the number of file-based secrets increases}
    \label{fig:overhead_attestation_agent_as_file_num_increasing}
\end{figure}

\begin{figure}[H]
    \centering
    \includegraphics[width=0.8\textwidth]{images/overhead_secret_injector_as_file_num_increasing.PNG}
    \caption[Benchmark result: Latency Overhead from Secret Injector as the number of file-based secrets increases]{Benchmark result: Latency Overhead from Secret Injector as the number of file-based secrets increases}
    \label{fig:overhead_secret_injector_as_file_num_increasing}
\end{figure}

% , which reveal the evolution of the overhead introduced by the remote attestation and provisioning agent, and 
% secret injector, as the number of file-type secrets grows.

\todo{ add lable after As discussed in Chapter 5,}
This paragraph discusses the benchmark results in Figure ~\ref{fig:overhead_attestation_agent_as_file_num_increasing} and Figure ~\ref{fig:overhead_secret_injector_as_file_num_increasing}. The results demonstrate that the increase in overhead from the remote attestation and provisioning agent, 
and secret injector, is proportional to the number of file-type secrets. As discussed in Chapter 5, constrained by the maximum TLS record size supported by the embedded\_tls library (16kb), the remote attestation and provisioning 
agent must perform one HTTP get + TLS operation for each file-type secret. For the secret injector, it is evident that as the number of files increases, it needs more time to inject the files into the target location in the application process.  Additionally,
Figure ~\ref{fig:startup_time_change_as_file_type_secret_increasing} reveals that although the overhead from the secret injector increases as the number of file secrets grows, it is negligible compared to the overhead from remote attestation and provisioning agent. Moreover, when the 
number of file-type secrets reaches 128, the overhead associated with the remote attestation and provisioning agents becomes astounding, reaching 4500 ms, accounting for over 70\% of the total program startup time.

\begin{figure}[H]
    \centering
    \includegraphics[width=0.8\textwidth]{images/startup_time_change_as_file_type_secret_increasing.PNG}
    \caption[Distribution of the time consumed by attestation \&\& provisioning agent, secret injector and others in the application startup]{Distribution of the time consumed by attestation \&\& provisioning agent, secret injector and others in the application startup}
    \label{fig:startup_time_change_as_file_type_secret_increasing}
\end{figure}



\subsubsection{Impact of measured Data Size}
This benchmark assesses the changes in the software measurer's latency overhead as the application binary size grows.

\begin{figure}[H]
    \centering
    \includegraphics[width=0.8\textwidth]{images/overhead_software_measurement_manager_as_elf_size_increasing.PNG}
    \caption[Benchmark result: Latency Overhead from Software Measurement Manager as the measured Data Size increases]{Benchmark result: Latency Overhead from Software Measurement Manager as the measured Data Size increases}
    \label{fig:overhead_software_measurement_manager_as_elf_size_increasing}
\end{figure}

\begin{figure}[H]
    \centering
    \includegraphics[width=0.8\textwidth]{images/startup_time_change_as_elf_size_increasing.PNG}
    \caption[Distribution of the time consumed by Attestation \&\& Provisioning Agent, Software Measurement Manager and others in the Application Startup]{Distribution of the time consumed by Attestation \&\& Provisioning Agent, Software Measurement Manager and others in the Application Startup}
    \label{fig:startup_time_change_as_elf_size_increasing}
\end{figure}



The benchmark results are in Figure ~\ref{fig:overhead_software_measurement_manager_as_elf_size_increasing}, which shows the latency overhead of the software measurer is proportional to the binary file size, i.e., the more data to be measured, the higher the overhead introduced by the 
software measurer. Furthermore, it is worth mentioning that when measured data less than or equal to 100 MiB, the latency overhead of the software measurer is lower than the overhead introduced by the remote attestation and provisioning agent. 
(Figure ~\ref{fig:startup_time_change_as_elf_size_increasing}).



\subsection{Macro-Benchmark– Application Life Cycle Performance Benchmark}\label{macri_app_start_up}

To evaluate the performance of confidential Quark in a real-world scenario, we conducted the following macro benchmark. The test compares the startup time, exit time, and runtime throughput of real-world applications in confidential and vanilla Quark using Nginx\cite*{nginx} and Redis\cite*{redis}. It should be noted that 
program startup time is measured from the start of the Qkernel to the completion of the dynamic linker (ld.so) loading the application's shared libraries, while the program exit time refers to the time from the time the exit system call is called until qkernel stops running. 
Regarding the application runtime throughput, we use the Apache HTTP server benchmarking tool (AB)\cite*{ab} and Redis-benchmark\cite*{Redis_benchmark} to generate workloads for Nginx and Redis, respectively


Regarding the benchmark methodology and setup, our testing framework\cite*{benchamark_framework} was extended to launch the target application repeatedly, record its startup and exit times, and results from AB\cite*{ab} and Redis-benchmark\cite*{Redis_benchmark}. Specifically, the test framework launches the 
target application 100 times, with the remote attestation and provisioning agent only fetching the shield policy from the relying party. For workload generator’s configuration, Redis benchmark\cite*{Redis_benchmark} and AB\cite*{ab} are set to issue 1,000,000 requests using 50 concurrent clients to Redis and Nginx, respectively. 
In addition, the size of the file requested by ab from Nginx\cite*{nginx} is 181 bytes.

\begin{figure}[H]
    \centering
    \includegraphics[width=0.8\textwidth]{images/reds_nginx_startup_comp.PNG}
    \caption[Redis \&\& Nginx Startup Time in Cquark vs vanilla Quark]{Redis \&\& Nginx Startup Time in confidential vs vanilla Quark}
    \label{fig:reds_nginx_startup_comp}
\end{figure}

The results presented in Figure ~\ref{fig:reds_nginx_startup_comp} indicate that confidential Quark takes about twice as long as vanilla quark to complete the application setup. This result is reasonable, given that Qkernel in Cquark must perform additional operations like remote attestation, 
secret provisioning, and host loaded data measurement, which prolongs the startup time of an application. When comparing the startup time of Nginx and Redis in the Cquark(~\ref{fig:reds_nginx_startup_comp}), we observed that Nginx takes approximately  324 ms longer than Redis to start. 
Since both applications only fetch the shield policy during startup, the latency overhead from the remote attestation and provisioning agent should be identical for both. Therefore, we deduce that the longer Nginx startup time is due to the software measurement manager measuring more data,
resulting in an increased startup time.

\begin{figure}[H]
    \centering
    \includegraphics[width=0.8\textwidth]{images/time_disribution_startup_redis_nginx.PNG}
    \caption[Redis \&\& Nginx  Startup Time in Cquark vs vanilla Quark]{Redis \&\& Nginx  Startup Time in Cquark vs vanilla Quark}
    \label{fig:time_disribution_startup_redis_nginx}
\end{figure}


\begin{table}[htbp]
    \centering
    \footnotesize

    \begin{tabularx}{1\textwidth}{@{} l L L L L L L@{}}
    \toprule
        Matrics    & ELF &  Shared Library   & Process Spec     & Guest Kernel Args  & Total Measurement\\
    \midrule

        \vn{Measurement for Redis}     &  3.64 MiB   &   12.61 MiB  &   304 Byte   &25004 Byte  & 16.27 MiB\\
        \vn{Measurement for Nginx}     &  5.5 MiB    &   61.64 MiB  &   304 Byte   &25004 Byte  & 67.16 MiB\\
    \bottomrule
    \end{tabularx}
    \caption{Software Measurement Manager measured Data during Application Startup\strut}
    \label{table: Measurement_For_Nginx_Redis}
\end{table}


To validate our hypothesis, we measured the overhead induced by the remote attestation and provisioning agent, the secret injector, the hardware evidence driver, and the software measurement manager, as well as the size of the data measured by the software measurer during the application 
startup in confidential Quark. The relevant details can be found in Figure  ~\ref{fig:time_disribution_startup_redis_nginx} and Table ~\ref{table:Measurement_For_Nginx_Redis}. Our findings show that the software measurer measured 16.27 MiB and 67.16 MiB of data for Redis and Nginx respectively, 
leading to a latency of 60.69 ms and 324 ms. Additionally, even though both applications have the same setup for the remote attestation and provisioning agent, the difference in latency overhead from the remote attestation and provisioning agent due to possible network transmission jitter 
was approximately 65 ms. Moreover, the latency generated from the hardware evidence driver and secret injection was less than 1 ms in both cases, which was insignificant compared to the software measurer and remote attestation and provisioning agent's overhead. To summarize, 
Nginx's startup time is longer compared to Redis due to two key reasons - network jitter and Nginx requires the software measurement manager to process more data during setup.

\begin{figure}[H]
    \centering
    \includegraphics[width=0.8\textwidth]{images/reds_nginx_exit_comp.PNG}
    \caption[Redis \&\& Nginx Exit Time in Cquark vs vanilla Quark]{Redis \&\& Nginx Exit Time in Cquark vs vanilla Quark}
    \label{fig:reds_nginx_exit_comp}
\end{figure}


With respect to the application (sandbox) exit, Figure ~\ref{fig:reds_nginx_exit_comp}  shows that confidential Quark can achieves the same performance as vanilla Quark. Specifically, in both confidential and vanilla Quark, the exit time of Redis and Nginx is around 2529 ms.

\begin{figure}[H]
    \centering
    \includegraphics[width=1\textwidth]{images/redis_throughput.PNG}
    \caption[Redis Throughout Test]{Redis Throughout Test Result}
    \label{fig:redis_throughput}
\end{figure}


\begin{figure}[H]
    \centering
    \includegraphics[width=1\textwidth]{images/nginx_throughput.PNG}
    \caption[Nginx Throughout Test]{Nginx Throughout Test Result}
    \label{fig:nginx_throughput}
\end{figure}


The throughput test results for Redis\cite*{redis} and Nginx\cite*{nginx} can be found in Figures ~\ref{fig:redis_throughput} and  ~\ref{fig:nginx_throughput}, respectively. Our analysis shows that execution of Redis and Nginx in confidential Quark led to a performance degradation of 
approximately 22\% and 10\%, respectively. The primary reason for this effect is the interception of the guest system calls.



\subsection{Trust Computing Base}\label{tcb}

In VM-based Trusted Execution Environments (TEE), the Trusted Computing Base (TCB) comprises the necessary hardware, firmware, and software modules, like the guest operating system, that ensure the required security guarantee while handling confidential data. The smaller the TCB, the lower the 
risk of a TEE being compromised by a vulnerability. Therefore, in this section, we evaluate the TCB overhead introduced by our implementation and compares the TCB size between confidential Quark and Confidential Container\cite*{confidential_kata}. To assess confidential Quark-incurred TCB overhead, 
we compare the lines of code and the size of the guest kernel binary in confidential and vanilla Quark as both use a application kernel as the guest kernel and the shield layer is part of  the guest kernel binary.  For the TCB size comparison between confidential Quark and the Confidential Container\cite*{confidential_kata}, 
we use the binary size of the Guest assets as a reference.  To ensure fairness, we employ the strip utility\cite*{strip} to remove unused symbols in binaries.

\begin{table}[htbp]
    \centering
\begin{tabular}{lrrrcrrr}\toprule
    \hline
    \multicolumn{3}{c}{Vanilla Quark}&&\multicolumn{3}{c}{Confidential Quark}\\
    \cline{1-3}\cline{5-7}
    $Component$ & $LoC$ & $Size$$^{3}$ && $Component$ & $LoC$ & $Size$$^{3}$\\
  \midrule
     Qkernel &32487 & -&& Qkernel&32487&-\\
     Qlib &85128  & -&&Qlib&  86609 &-\\
     Shield  & 0 & - && Shield&2805&-\\
  
     \midrule
      Total  & 117615& 3.1MiB&& & 121901 &  4.6 MiB\\
      \bottomrule
  \end{tabular}

  \caption{Comparison of the LoC and compiled size of vanilla quark and confidential quark\strut}
  \label{table:tcb_size_quark_vs_cquark}
\end{table}

Table ~\ref{table:tcb_size_quark_vs_cquark} provides a summary of the TCB overhead required to achieve confidentiality in confidential Quark. The vanilla Quark version v2.0 has a guest kernel with 11,7615 lines of code and generates a 3.10 MB static linked binary. In confidential Quark, our implementation adds 4286 LoC to the guest binary, 
bringing the total to 12,1901 LoC and 4.6 MB of compiled static linked binary. Overall, our implementation makes the guest binary 1.48 times larger compared to vanilla Quark.


\begin{table}[htbp]
    \centering
\begin{tabular}{lrrcrr}\toprule
    \hline
    \multicolumn{2}{c}{Confidential Container}&&\multicolumn{2}{c}{Confidential Quark}\\
    \cline{1-2}\cline{4-5}
    $Component$  & $Size$$^{3}$ && $Component$  & $Size$$^{3}$\\
  \midrule
     Guest Kernel &47.16 MiB && Guest kernel &4.6 MiB\\
     OVMF   & 4 MiB && &   &\\
     Kata agent  & 22.65MiB  && &&\\
  
     \midrule
        & 73.81 MB&& &   4.6 MiB\\
      \bottomrule
  \end{tabular}

  \caption{TCB size comparison between confidential container vs confidential quark in terms of compiled size of guest assets \strut}
  \label{table:tcb_size_quark_vs_kata}
\end{table}

Table ~\ref{table:tcb_size_quark_vs_kata} shows the outcomes of comparing the TCB size in the Confidential Container (Kata)\cite*{confidential_kata} and confidential Quark. Unlike confidential Quark, which uses the application kernel as the guest kernel, Confidential Container deploys 
the qemu-linux base virtual machine. Therefore, the guest assets in the Confidential Container context include the Kata agent\cite*{kata_agent}, OVMF\cite*{ovmf}, and Linux Guest kernel. In the Confidential Container version v0.4, the compiled static binary sizes of Kata agent, OVMF, and Guest kernel are 22.65 Mb, 4.0 MB, and 47.16 MB, 
respectively. This totals a TCB size of 73.81 MB, which is about 16 times greater than that of confidential Quark.

It is noteworthy that there is potential to further optimize the size of Quark's Trusted Computing Base (TCB).  In Quarkv version 2.0, qlib contributes some redundant code to guest kernel binary. To facilitate code development, the qlib code is shared between qvisor and guest kernel. 
However, our founding shows some code in qlib is only used by functions in qvisor, thus increasing the size of the guest kernel’s binary. 

\section{Summary}
In this chapter, I conducted both qualitative and quantitative analyses to evaluate my work. In the qualitative analysis, I identified potential security vulnerabilities in vanilla Quark and explained how my work mitigates these vulnerabilities. 
Regarding the quantitative analysis, I developed a testing framework to characterize the performance of CQuark. The obtained results showed that confidential Quark engenders significant overhead compared to vanilla Quark. 
This overhead is visible in longer program creation times, reduced runtime program throughput, reduced speed of issuing instructions to the program, increased trusted computing base (TCB) size, etc.




% \begin{table}[htp]
%     \caption{Software measurement manager measured data during application startup}
%     \begin{tabularx}{\textwidth}{*{2}{p{0.5\textwidth}}}\toprule
%                                  & ELF &  Shared Library   & Process Spec  & Stack   & Guest Kernel Args  & Total Measurement \\ \midrul
%     Measurement For Hello world  &  1.4468 MiB    &   0 MiB  &   3819 Byte   & 305 Byte   &5865 Byte  & 1.457 MiB \\\addlinespace
%     \bottomrule
%     \end{tabular}
%     \label{table: Measurement_For_Hello_world}
% \end{table}


% \begin{table}[!ht]
%     \sffamily
%     \caption{Terminologies in SQL \& corresponding in MongoDB}
%     \label{tab:my_label}
%     \centering
%     \begin{tabularx}{\textwidth}{*{2}{p{0.5\textwidth}}}
%     \toprule
%                                  & ELF &  Shared Library   & Process Spec  & Stack   & Guest Kernel Args  & Total Measurement \\ \midrule
%     Measurement For Hello world  &  1.4468 MiB    &   0 MiB  &   3819 Byte   & 305 Byte   &5865 Byte  & 1.457 MiB \\\addlinespace
%     \dots & \dots \\
%     \bottomrule
%     \end{tabularx}
%     \end{table}




\cleardoublepage

%%% Local Variables:
%%% TeX-master: "diplom"
%%% End:
