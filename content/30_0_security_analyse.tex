\chapter{Security Analyze}
\label{sec:security_analyse}

% Ist das zentrale Kapitel der Arbeit. Hier werden das Ziel sowie die
% eigenen Ideen, Wertungen, Entwurfsentscheidungen vorgebracht. Es kann
% sich lohnen, verschiedene Möglichkeiten durchzuspielen und dann
% explizit zu begründen, warum man sich für eine bestimmte entschieden
% hat. Dieses Kapitel sollte - zumindest in Stichworten - schon bei den
% ersten Festlegungen eines Entwurfs skizziert werden.
% Es wird sich aber in einer normal verlaufenden
% Arbeit dauernd etwas daran ändern. Das Kapitel darf nicht zu
% detailliert werden, sonst langweilt sich der Leser. Es ist sehr
% wichtig, das richtige Abstraktionsniveau zu finden. Beim Verfassen
% sollte man auf die Wiederverwendbarkeit des Textes achten.

% Plant man eine Veröffentlichung aus der Arbeit zu machen, können von
% diesem Kapitel Teile genommen werden. Das Kapitel wird in der Regel
% wohl mindestens 8 Seiten haben, mehr als 20 können ein Hinweis darauf
% sein, daß das Abstraktionsniveau verfehlt wurde.

%\ldots design \ldots

%\todo{write design}

This chapter aims to analyze the potential vulnerabilities in Vanilla Quark, through which an adversary can obtain access to confidential information of an application. The first section defines the threat model(Section 3.1). Subsequently, this chapter scrutinizes the Vanilla Quark's security from the perspective of the Open Container Initiative (OCI) interface (Section 3.2).



\section{Threat model}

We employs the OWASP application threat modeling methodology\cite*{OWASP_Threat_Modeling} to create a threat model comprising four aspects: Actors, assets , external dependencies, and attack surface.

\textbf{Actors}. Our model involves the cloud provider and the tenant. While the cloud provider is accountable for providing the hardware and software necessary for running and orchestrating applications, the tenant distrusts the cloud provider in the context of a confidential computing environment. Therefore, any software components within the cloud infrastructure - 
Kubernetes control plane\cite*{k8s}, containerd\cite*{containerd}, and hypervisor are untrusted.

\textbf{Assets}. Our objective is to secure the Kubernetes workload(application) itself, i.e., preserving the integrity of the application binary and its dependencies, maintaining the confidentiality and integrity of the data generated by the application during runtime, as well as protecting the secrets provided to the application by its owner.

\textbf{External Dependencies}. We assume that the Guests, including Qkernel and Kubernetes workloads, operate within an encrypted virtual machine to prevent malicious hypervisor(Qvisor) or powerful cloud operator from accessing the workload’s sensitive data through guest memory and registers. Furthermore, we exclude attacks related to denial of service\cite*{DOS_ATTACK}, side channell\cite*{217454}, 
network, or file systems.

\textbf{Attack Surface}. We primarily focus on attacks occurring on the OCI interface\cite*{oci-runtime-spec}. As the management of k8s workloads is the responsibility of the cloud provider, it must have a means of accessing workloads for orchestration, even if the workloads are running within a TEE. Although the OCI runtime specification offers this possibility to the cloud provider, 
it exposes a new attack surface for an adversary to probe k8s workloads’ secrets. Additionally, since the implementation of OCI in the guest interacts with the host through the Hypercall, Qcall, or Ucall interface, we also considered attacks from this interface.



\cleardoublepage

%%% Local Variables:
%%% TeX-master: "diplom"
%%% End:
