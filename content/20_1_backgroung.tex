\chapter{Technical Background}
\label{sec:state}

% Hier werden zwei wesentliche Aufgaben erledigt:

% 1. Der Leser muß alles beigebracht bekommen, was er zum Verständnis
% der späteren Kapitel braucht. Insbesondere sind in unserem Fach die
% Systemvoraussetzungen zu klären, die man später benutzt. Zulässig ist
% auch, daß man hier auf Tutorials oder Ähnliches verweist, die hier auf
% dem Netz zugänglich sind.

% 2. Es muß klar werden, was anderswo zu diesem Problem gearbeitet
% wird. Insbesondere sollen natürlich die Lücken der anderen klar
% werden. Warum ist die eigene Arbeit, der eigene Ansatz wichtig, um
% hier den Stand der Technik weiterzubringen? Dieses Kapitel wird von
% vielen Lesern übergangen (nicht aber vom Gutachter ;-), auch später
% bei Veröffentlichungen ist "Related Work" eine wichtige Sache.

% Viele Leser stellen dann später fest, daß sie einige der Grundlagen
% doch brauchen und blättern zurück. Deshalb ist es gut,
% Rückwärtsverweise in späteren Kapiteln zu haben, und zwar so, daß man
% die Abschnitte, auf die verwiesen wird, auch für sich lesen
% kann. Diese Kapitel kann relativ lang werden, je größer der Kontext
% der Arbeit, desto länger. Es lohnt sich auch! Den Text kann man unter
% Umständen wiederverwenden, indem man ihn als "Tutorial" zu einem
% Gebiet auch dem Netz zugänglich macht.

% Dadurch gewinnt man manchmal wertvolle Hinweise von Kollegen. Dieses
% Kapitel wird in der Regel zuerst geschrieben und ist das Einfachste
% (oder das Schwerste weil erste).

%\ldots state of the art \ldots

%\todo{write state}
\section{Kubernetes}
Kubernetes\cite*{k8s} is a portable, scalable open-source platform for managing and orchestrating containerized workloads. Kubernetes emerged because, in the context of the rise of cloud computing, more and more applications are being deployed to the cloud, 
and cloud deployers need an automated tool to manage thousands of containers. Kubernetes provides deployers with a framework for running cloud-native applications\cite*{KRATZKE20171} in both elastically and resiliently manner. 
It undertakes application scaling and failover and offers a variety of deployment models.

Kubernetes also introduces the concept of pod. In a cluster, a pod, which consists of one or more containers sharing network and storage resources, is the smallest deployment unit managed by Kubernetes. It has a unique cluster-IP address, and the containers within it 
can communicate with each other through the localhost. In addition, Kubernetes support different isolation levels of pods. With the runtime class keyword, clients can choose to deploy pods based on Linux namespace and cgroup for better performance, 
or hardware virtualization for better security\cite*{k8s_runtime_class}.

\subsection{Kubernetes architecture}
\todo{add k8s architectural diagram}

From an architectural point of view, a k8s cluster consists of at least one master node and several worker nodes. A node can be a physical or virtual machine. The master node consists of a series of control plane components that make global decisions 
about the cluster, including scheduling, detecting, and responding to cluster events, for example, dynamically adjusting the number of 
pods in an application based on resource utilization. As shown in the diagram, these control plane components include the kube-apiserver, kube-controller, kube-scheduler, and ectd\cite*{k8s}. Kube-apiserver provides the kubernetes control plane with an 
interface to control and manage the cluster. Specifically, it can be used to receive and authenticate requests from users and other components of the cluster, and update the corresponding API objects that include services pods, and replication 
controllers. Kube-controller is a control loop that continuously monitors the state of the cluster and attempts to move the current state of the cluster to the desired state. For example, the Node controller is responsible for monitoring the state 
of each node and responding accordingly when a node crash. Kube-scheduler continuously search undeployed pods and deploys them to an appropriate worker node. Factors that influence scheduling decisions are resource requirements, 
hardware/software/policy constraints, affinity, data locality, and inter-workload interference. The etcd is a distributed, highly available key-value store for storing cluster level data.

The worker node, which is managed by the master node runs and maintains the pods. It consists of three main components, kubelet, container runtime (like containerd\cite*{containerd} and ciro\cite*{cri-o}), and kube-proxy. Kubelet is responsible for communication between the 
worker node and the master node and for ensuring that the pods runs properly on the worker node. Container runtime is used to create and manage image and  container lifecycle, which includes pulling images from the remote registry, unpacking images, 
initializing the container's runtime environment (setting up storage, network), and running the container. The kube-proxy running on each node implements part of the Kubernetes service and maintains network rules on each node, which allows 
network access to pods from both within and outside the cluster

\subsection{Open Standards (CRI\cite*{cri-interface}, OCI\cite*{oci-spec}) makes Kubernetes more extensible}

CRI introduces an abstraction layer that empowers kubelet to work with a diverse range of container runtimes without the need to modify cluster components. It defines the requirements of the 
orchestration system regarding the container runtime and decouples the container runtime from the kubelet source code, which dramatically reduces the effort involved in integrating container 
runtimes into k8s\cite*{cri-interface}.

\todo{cri overview figure}

Figure 3 shows the overview of CRI. kubelet communicates with the container runtime via sockets, where the container runtime implements a CRI plugin as a socket server according to the CRI 
requirements. The socket server provides two primary services to kubelet, namely ImageService and RuntimeService. The ImageService offers kubelet services such as pulling images from remote and 
inspecting and deleting images managed by the container runtime. The RuntimeService, on the other hand, enables the kubelet to manage the container and sandbox lifecycle 
(RunPodSandbox /StopPodSandbox/RemovePodSandbox/PodSandboxStatus/CreateContainer/StartContainer/StopContainer, etc.) 
and interact with containers (attach/port-forward/exec)\cite*{cri-interface}.



\subsection{Loging}


\section{AMD SEV}
\subsection{History of AMD SEV}
\subsection{MEMORY PROTECTION}
\subsection{Non-Automatic VM Exits}
\subsection{REMOTE ATTESTATION}

\section{QUARK CONTAINER}

\section{ACCESS CONTROL}


\cleardoublepage

%%% Local Variables:
%%% TeX-master: "diplom"
%%% End:
